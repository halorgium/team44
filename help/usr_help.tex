\documentclass{article}
\usepackage{}
\title{User's Guide to MusicLib (MusicLib for Dummies)}
\author{Daniel Bakker}

\begin{document}
\maketitle
\tableofcontents
\newpage
\section{Introduction}
The Musiclib program is designed to support a library of music that can be accessed by registered users via the web. MusicLib allows users to find different albums within the database and take them out on loan for a period of time. Users can also read and write comments about the different artists and albums that are of interest.

\section{How To Use}
MusicLib is run through your web browser, and all you need to do is click on the desired links to access the functionality of the program. You must decide on a user name for yourself when you first log in. This allows the system to identify every different user in the system. If there are any problems with using the system emails can be sent to us (the programmers) via the ``Contact Us'' link. 
\section{Getting Started}
If it is your first time using the system, you can enter the name ``nairarbil'' as your login identity. From the ``nairarbil'' account you may add other users as required and access all the MusicLib functions. 

\section{User Functionality \& Site Navigation}
While logged in, users can do a variety of different things. Most of these are accessed from the links in the box to the left of the browser window.
\begin{description}
\item{\bf{Home}}
This takes you to the homepage of the Music Library. Displayed on this page is the introductory information about our music library.
\item{\bf{News}}
Clicking this link takes you to the page where the latest news and updates are displayed.
\item{\bf{Contact Us}}
This is where you can find the email addresses of the programmers so you can complain if there are any problems, or compliment if you like the service. Clicking any of the email addresses will automatically open up an email window if you have this configured in your browser.

\item{\bf{Logout}}
To exit the system, you must click on this link. This tells the program that you are no longer within the system.
\item{\bf{My Account}}
This displays your current user information, including whether or not you have Librarian privileges.
\item{\bf{My Borrowing History}}
Brings up a page showing all the albums that you have currently on loan, or have loaned in the past. This is where you can return your albums once you have finished with them, by clicking on the ``return album'' button.
\item{\bf{My Comments}}
Here is where you can view all the comments that you have made about different artists and albums in the past. You can also see any comments that Librarians have written about you as a user.
\item{\bf{View All Artists}}
This page shows a list of all the Artists that are currently available in the database. From the list you can select an artist of interest and view their details, or write a comment about them.

\item{\bf{View All Albums}}
This link brings up a page showing all the Albums currently in the database. It also shows if they are on loan or currently available for borrowing.
\end{description}

\section{Librarian Functionality}
Librarians are a special type of user. If you have been given librarian status, you have access to all the user functionality, as well as the following;
\begin{description}
\item{\bf{Add New Artist}}
This allows Librarians to put new artists in the database.
\item{\bf{Add New Album}}
The same as Add New Artist except for albums.
\item{\bf{Add New User}}
This page lets Librarians create user accounts for new people.
\item{\bf{View All Users}}
This Page brings up a list of all the users currently registered in the database. There is a separate table for normal users and librarians.
\end{description}
\section{Extra Help}
If more detailed help is required, or you are interested in the actual code that is used, please read the programmers help document.


\end{document}
